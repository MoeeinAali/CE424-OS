\documentclass[10pt,a4paper]{article}
\usepackage{tocloft}
\usepackage{commons/course}
\usepackage{listings}
\usepackage{xcolor}

% Define colors for syntax highlighting
\definecolor{keywordstyle}{rgb}{0.0, 0.4, 0.8} % A bluer shade
\definecolor{stringstyle}{rgb}{0.9, 0.17, 0.31} % amaranth
\definecolor{commentstyle}{rgb}{0.1, 0.6, 0.2} % green
\definecolor{numberstyle}{rgb}{0.4, 0.4, 0.4} % gray
\definecolor{backgroundcolor}{rgb}{0.94, 0.97, 1.0} % aliceblue

% Define the Python-like language for syntax highlighting
\lstdefinelanguage{PythonLike}{
	morekeywords={mov, add, sub, cmp, jmp , lw , addi,sw,bne},
	sensitive=false,
	morecomment=[l]{;},
	morestring=[b]",
}

% Set the style for Python-like code
\lstset{
	language=PythonLike,
	basicstyle=\ttfamily\small,
	keywordstyle=\color{keywordstyle},
	stringstyle=\color{stringstyle},
	commentstyle=\color{commentstyle},
	numbers=left,
	numberstyle=\tiny\color{numberstyle},
	stepnumber=1,
	numbersep=5pt,
	keepspaces=true,
	tabsize=4,
	showspaces=false,
	showstringspaces=false,
	showtabs=false,
	breaklines=true,
	breakatwhitespace=false,
	frame=single,
	backgroundcolor=\color{backgroundcolor},
}

\newlistof{listofproblems}{lop}{\normalsize فهرست مسائل}
\newcommand{\addproblem}[1]{\addcontentsline{lop}{listofproblems}{\protect\numberline{}#1}}


\begin{document}


\سربرگ{فاز اول پروژه}{}{پاسخ‌دهنده: معین آعلی - 401105561}{استاد: رسول جلیلی}


\section{مقدمه‌ای بر پروژه}

پروژه‌ای که قرار است توسعه داده شود، مربوط به طراحی و پیاده‌سازی یک سیستم مدیریت و اجرای محفظه‌ها (Containers) مشابه سیستم‌های شناخته‌شده‌ای همچون Docker یا Podman است. این پروژه، بدون نیاز به معماری مبتنی بر دیمون (Daemonless) اجرا می‌شود که باعث افزایش پایداری و جلوگیری از نقطه تکی شکست (Single-point-of-failure) خواهد شد.

هدف اصلی این پروژه، ایجاد و مدیریت محیط‌های منزوی برای اجرای برنامه‌هاست که از طریق تکنولوژی‌هایی همچون namespace ، cgroups ، chroot و union filesystem محقق می‌شود. همچنین از تکنولوژی eBPF برای نظارت عمیق‌تر بر تعامل برنامه‌ها با سیستم عامل لینوکس استفاده خواهد شد.


\section{اهداف و ویژگی‌های پروژه}

\subsection{ایجاد محیط‌های منزوی (Isolation)}

\begin{itemize}
	\item
	 استفاده از namespace ها جهت تفکیک فضای PID ، UID ، GID ، Hostname و فایل سیستم از محیط میزبان.
	\item محدود کردن دسترسی به فایل‌ها و دایرکتوری‌ها با
	 chroot .
	\item
	 پیاده‌سازی سیستم فایل Union-filesystem (مانند overlayfs ) برای جداسازی پویا و ترکیب دایرکتوری‌ها.
\end{itemize}

\subsection{مدیریت منابع با استفاده از cgroups}

\begin{itemize}
	\item مدیریت منابع شامل حافظه، پردازنده، شبکه و I/O برای هر محفظه.
	\item کنترل و اعمال محدودیت‌های دقیق بر منابع اختصاص‌یافته به هر محفظه.
\end{itemize}


\subsection{تعامل با هسته لینوکس}

\begin{itemize}
	\item 
	استفاده از قابلیت eBPF برای نظارت بر فراخوانی‌های سیستمی و ایجاد رفتار پویا در فضای هسته.
\end{itemize}


\subsection{ارتباط بین محفظه‌ها}

\begin{itemize}
	\item 
	پشتیبانی از
	 Inter-process-communication(IPC)
	 برای ارتباط بین محفظه‌ها در شرایط خاص.
	
	\item 
	امکان استفاده از mount های اشتراکی (propagate) جهت انتشار mount ها بین محفظه‌ها.
\end{itemize}

\pagebreak

\subsection{کنترل و مدیریت اجرای محفظه‌ها}

\begin{itemize}
	\item 
	ایجاد رابط کاربری ساده و کاربردی با فرمان‌هایی نظیر run، start، list و status برای مدیریت محفظه‌ها.
	\item 
	قابلیت ایجاد، متوقف‌سازی، راه‌اندازی مجدد و حذف محفظه‌ها.
\end{itemize}



\section{قابلیت‌های اضافی}

\begin{itemize}
	\item
	 پیاده‌سازی قابلیت توقف (freeze) محفظه‌ها و ادامه دادن کار از نقطه متوقف شده، با استفاده از cgroups-freezer .
	
	\item 
	ایجاد قابلیت‌هایی شبیه Docker-images برای استفاده از سیستم فایل Union-filesystem .
\end{itemize}


\section{ابزارها و زبان‌های برنامه‌نویسی پیشنهادی}


\subsection{زبان برنامه‌نویسی پیشنهادی:}

\begin{itemize}
	\item
	 Go (Golang) :
	  پیشنهاد اول به دلیل پشتیبانی قوی از قابلیت‌های سطح پایین لینوکس، وجود کتابخانه‌های آماده برای namespace و cgroups ، و استفاده گسترده در ابزارهای مشابه Docker و Kubernetes و Podman .
	
	\item 
	Rust :
	 پیشنهاد دوم برای اطمینان از ایمنی حافظه و تعامل قدرتمند با سیستم عامل.
\end{itemize}

\subsection{ابزارهای مورد نیاز:}

\begin{itemize}
	\item 
	Linux-Kernel (هسته لینوکس): ضروری برای استفاده از namespace و cgroups .
	
	\item 
	eBPF-Toolchain: شامل BCC یا libbpf برای پیاده‌سازی بخش‌های نظارتی پروژه.
	
	\item 
	OverlayFS : جهت پیاده‌سازی Union-Filesystem .
	
	\item 
	systemd و ابزارهای مرتبط: برای مدیریت پردازش‌ها (اختیاری اما کاربردی در محیط‌های لینوکسی).
\end{itemize}


\section{پیش‌نیازهای فنی پروژه}
برای اجرای موفق این پروژه نیاز به پیش‌نیازهای زیر دارید:
\begin{itemize}
	\item
	 دانش قوی از ساختار سیستم عامل لینوکس به خصوص namespace ، cgroups و chroot .
	\item
	 درک کافی از مدیریت منابع سیستم و تعامل با هسته لینوکس.
	\item
	 آشنایی با اصول طراحی سیستم‌های Daemonless .
	\item
	 تسلط نسبی به eBPF برای ایجاد قابلیت نظارتی قدرتمند.
	\item
	 آشنایی با مفاهیم پایه‌ای شبکه و IPC .
\end{itemize}


\section{توضیحات اهداف پروژه}

\subsection{آشنایی با Namespace ها :}

Namespace در لینوکس ابزاری است که منابع مختلف سیستم (مثل PID فرایندها، شبکه، کاربران، فایل سیستم و...) را از دید برنامه‌ها جداسازی کرده و به هر برنامه محیطی مجزا و منزوی ارائه می‌دهد.


\subsection{مدیریت منابع با cgroups :}

Control-Groups یا cgroups به شما امکان می‌دهد میزان استفاده از منابع سیستم را برای هر محفظه یا برنامه محدود کرده و دقیقاً کنترل کنید تا مصرف بی‌رویه منابع توسط برنامه‌ها رخ ندهد.


\subsection{منزوی‌سازی محیط‌ها با chroot :}
Chroot ابزاری است که به کمک آن دسترسی یک برنامه به سیستم فایل اصلی محدود می‌شود؛ درواقع برنامه در محیطی محدود اجرا می‌شود و نمی‌تواند به فایل‌ها و دایرکتوری‌های خارج از آن دسترسی پیدا کند.

\subsection{ساخت سیستم‌های فایل Union :}
Union-Filesystem ( مانند overlayfs ) روشی برای ترکیب چند دایرکتوری به صورت پویا است که به ایجاد فایل‌سیستم‌های جداگانه برای هر محفظه کمک می‌کند؛ به این ترتیب هر محفظه یک دید متفاوت و مجزا به فایل‌ها دارد.

\subsection{ایجاد یک سیستم مدیریت محفظه:}
هدف این است که سیستمی مشابه ابزارهایی مانند Podman طراحی شود که وظیفه ایجاد، مدیریت، اجرا و نظارت بر محفظه‌ها را دارد و یک رابط ساده برای کنترل آن‌ها ارائه می‌دهد.

\subsection{تعامل با سیستم‌عامل لینوکس:}
در این هدف لازم است مستقیماً با هسته لینوکس ارتباط برقرار کرده و با استفاده از فراخوانی‌های سیستمی و کدنویسی سطح پایین، قابلیت‌های عمیق‌تری در سیستم ایجاد شود.

\subsection{آشنایی با eBPF :}
eBPF تکنولوژی قدرتمندی است که اجازه می‌دهد رفتار دلخواه و پویایی در هسته سیستم‌عامل ایجاد کرده و به شکل مؤثر و ایمن فعالیت‌های سیستم عامل و برنامه‌ها را مانیتور یا کنترل کنید.


\end{document}